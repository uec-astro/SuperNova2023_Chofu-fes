%%% OPTIONAL PACKAGE %%%
% \usepackage{bm} % ベクトル
% 
% \usepackage{physics2} % 物理計算式の簡便化マクロ 
% physics -> physics2 https://qiita.com/Yarakashi_Kikohshi/items/131e2324f401c3effb84
% \usepackage{circuitikz} % 回路系の描画用マクロ
% 
% \usepackage[version=4]{mhchem}
% 
% \usepackage{ulem} % 下線・波線・打ち消し線をつける
% 
% \usepackage{framed} % 囲み付き文章を出すためのパッケージ
% \usepackage{type1cm} % 文字の大きさを自由に変えるためのパッケージ
% \usepackage{caption} % キャプションとサブキャプションのパッケージ
% \usepackage{subcaption}
% \usepackage{enumitem} %箇条書きのカスタマイズが可能
% \usepackage{enumerate} % 高機能番号付き箇条書きのパッケージ
% \usepackage{paralist} % インラインリストのパッケージ
%\usepackage{minted} % ソースコード表示の用パッケージ
% cf) https://qiita.com/float168/items/2884a4d80a54ffa89a34
%\usepackage{mdframed} % ページを跨ぐソースコード用
%
%\usepackage{sansmathfonts} % 数式フォントについて
%\usepackage{textcomp} % 特殊文字
%\usepackage{mathcomp} % 数学用フォント
%\usepackage[hyphens]{xurl} % URLを表示するためのパッケージ、ハイフンでの改行を許可
%%%%%%%%%%%%%%%%%%%%%%%%%